
\section{Proofs} \label{s:proofs}

\subsection{Theorem \ref{thm: Steenrod-Adem on Barratt-Eccles}}

We prove that the $\Z[\mathrm{C}_r]$-linear map
\begin{equation*}
\psi_{\mathcal E}(r) : \mathcal W(r) \to \mathcal E(r)
\end{equation*}
introduced in Definition \ref{def: Steenrod-Adem on Barratt-Eccles} is a chain map for every $r \geq 0$. We simplify notation and write $\psi$ instead of $\psi_{\mathcal E}(r)$. To show $\psi$ is a chain map we proceed by induction. Notice that
\begin{equation*}
\psi(\partial e_0) = 0 = \partial \psi(e_0)
\end{equation*}
and assume $\psi(\partial e_{k-1}) = \partial \psi(e_{k-1})$. If $k = 2n$ we have
\begin{align*}
\partial \psi(e_{2n}) & = 
\partial \sum_{r_1, \dots, r_n} 
\big(\rho^0, \rho^{r_1}, \rho^{r_1+1}, \dots, \rho^{r_n}, \rho^{r_n+1} \big)  \\ & =
\partial \sum_{r_2, \dots, r_n} \sum_{r_1 = 0}^{p-1}
\big(\rho^0, \rho^{r_1} \, (\rho^0, \rho^{1}, \dots, \rho^{r_n-r_1}, \rho^{r_n - r_1 +1}) \big) \\ & =
\partial \sum_{r_2, \dots, r_n}
\big(\rho^0, N\, (\rho^{0}, \rho^{1}, \dots, \rho^{r_n}, \rho^{r_n + 1}) \big) \\ & =
\sum_{r_2, \dots, r_n}
N\, \big( \rho^{0}, \rho^{1}, \dots, \rho^{r_n}, \rho^{r_n + 1} \big) \\ & -
\sum_{r_2, \dots, r_n}
\big(\rho^0, \partial \, N \, (\rho^{0}, \rho^{1}, \dots, \rho^{r_n}, \rho^{r_n+1}) \big) \\ & =
N \psi(e_{2n-1}) - (\rho^0, \partial N \psi (e_{2n-1})) \\ & =
\psi(N e_{2n-1}) - (\rho^0, \psi (\partial N e_{2n-1})) \\ & =
\psi(\partial e_{2n}) - (\rho^0, \psi (\partial^2 e_{2n})) \\ & =
\psi(\partial e_{2n}).
\end{align*}
If $k = 2n+1$ we have
\begin{align*}
\partial \psi(e_{2n+1}) & = 
\partial \sum_{r_1, \dots, r_n} 
\big(\rho^0, \rho^1, \rho^{r_1}, \rho^{r_1+1}, \dots, \rho^{r_n}, \rho^{r_n+1} \big)  \\ & =
\partial \sum_{r_1, \dots, r_n}
\big(\rho^0, \rho^{1} \, (\rho^0, \rho^{r_1-1}, \rho^{r_1}, \dots, \rho^{r_n - 1}, \rho^{r_n}) \big) \\ & =
\partial \sum_{r_1, \dots, r_n}
\big(\rho^0, T\, (\rho^0, \rho^{r_1-1}, \rho^{r_1}, \dots, \rho^{r_n - 1}, \rho^{r_n}) \big) \\ & =
\sum_{r_1, \dots, r_n}
T \, \big( \rho^0, \rho^{r_1-1}, \rho^{r_1}, \dots, \rho^{r_n - 1}, \rho^{r_n} \big) \\ & -
\sum_{r_1, \dots, r_n}
\big(\rho^0, \partial \, T \, (\rho^0, \rho^{r_1-1}, \rho^{r_1}, \dots, \rho^{r_n - 1}, \rho^{r_n}) \big) \\ & =
T \psi(e_{2n}) - (\rho^0, \partial T \psi (e_{2n})) \\ & =
\psi(T e_{2n}) - (\rho^0, \psi (\partial T e_{2n})) \\ & =
\psi(\partial e_{2n+1}) - (\rho^0, \psi (\partial^2 e_{2n+1})) \\ & =
\psi(\partial e_{2n+1})
\end{align*}
where for the third equality we used that for any $r_1, \dots, r_n$
\begin{equation*}
(\rho^0, \rho^0, \rho^{r_1-1}, \rho^{r_1}, \dots, \rho^{r_n - 1}, \rho^{r_n}) = 0.
\end{equation*}
This map is a quasi-isomorphism since both complexes have the homology of a point and $\psi(e_0)$ represents a generator of the homology.

\subsection{Theorem \ref{thm: Steenrod-Adem on surjection MS convention}}

\begin{proof}
	We prove that the $\Z[\mathrm{C}_r]$-linear map
	\begin{equation*}
	\psi_{\mathcal X}(r) : \mathcal W(r) \to \mathcal X(r)
	\end{equation*}
	introduced in Definition \ref{def: Steenrod-Adem on surjection} is a chain map for every $r \geq 0$. We simplify notation and write $\psi$ instead of $\psi_{\mathcal X}(r)$. To show $\psi$ is a chain map we proceed by induction. Notice that
	\begin{equation*}
	\psi(\partial e_0) = 0 = \partial \psi(e_0)
	\end{equation*}
	and assume $\psi(\partial e_{n-1}) = \partial \psi(e_{n-1})$. For $n = 2m+1$ we have
	\begin{align*}
	\partial \psi(e_{2m+1}) 
	& =
	\partial\, h\, T\, \psi(e_{2m}) \big) \\
	& =
	T\, \psi(e_{2m}) - i^{r-1} p^{r-1}\, \psi(e_{2m}) -
	h\, \partial\, T\, \psi(e_{2m}) \big) \\
	& =
	T\, \psi(e_{2m}) - 
	h\, T\, \psi(\partial\, e_{2m}) \big) \\
	& =
	T\, \psi(e_{2m}) - 
	h\, \psi(T\,N\, e_{2m-1}) \big) \\
	& = 
	T \psi(e_{2m}).
	\end{align*}
	For $n = 2m$ the proof is analogous. The chain map $\psi$ is a quasi-isomorphism since both complexes have the homology of a point and $\psi(e_0) = (1, \dots, r)$ is a generator of the homology.
\end{proof}